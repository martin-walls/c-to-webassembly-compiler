\documentclass[00-main.tex]{subfiles}

\begin{document}

\chapter{Introduction}

\begin{mrwComment}
Word budget: \textasciitilde500--600 words
\end{mrwComment}

\section{Background and Motivation}

Increasingly in modern society, more and more applications are shifting to cloud computing as one of the primary ways of interacting with computers.
We are experiencing a transition away from traditional native applications and towards performing the same tasks in an online environment.
However, the standard approach of building web apps with JavaScript fails to deliver the performance necessary for many intensive applications.

WebAssembly (abbreviated Wasm) aims to solve this problem by bringing near-native performance to the browser space.
It is a virtual instruction set architecture, which executes on a stack-based virtual machine.
Per the Introduction section of the WebAssembly specification \ccite{wasm-spec}, it is designed to have ``fast, safe, and portable semantics'' and an ``efficient and portable representation''.
The next two paragraphs expand on what this entails.

The semantics are designed to be able to be executed efficiently across different hardware, be memory safe (with respect to the surrounding execution environment), and to be portable across source languages, target architectures, and platforms.

The representation is designed with the primary target of the web in mind.
It is designed to be compact and modular, allowing it to be efficiently transmitted over the Internet without slowing down page loads.
This also includes being streamable and parallelisable, which means it can be decoded while still being received.
% It supports both just-in-time and ahead-of-time compilation.

This project creates a compiler from C to WebAssembly.
I chose C as a source language because it is quite a low-level language, which allowed me to have time to work on compiler optimisations rather than just implementing language features.
In particular, C has manual memory management, therefore I did not implement a garbage collector or similar.


\section{Survey of Related Work}

Many other compilers to WebAssembly exist, covering a large number of source languages~\ccite{awesome-wasm-langs}.
Notably, Emscripten~\ccite{emscripten-homepage} is a compiler from LLVM~\ccite{llvm} to WebAssembly and JavaScript.
This primarily targets C and C++, however many other languages also have LLVM compilers.

Cheerp~\ccite{cheerp} is an alternative C/C++ to WebAssembly compiler, which can produce a combination of WebAssembly and JavaScript, and prides itself on generating very optimised code.

Compilers to WebAssembly exist for other mainstream languages such as Java, C\#, Python, and even JavaScript.
It is an area of active development; many more WebAssembly compilers are in progress for other languages.

My project covers a subset of the existing C to WebAssembly compilers; I only support a subset of the language features, and focus on correctness over optimality.
I have implemented some optimisations that provide significant performance improvements to many programs, however not to the extent of optimising every possible edge case as some industry compilers do.

\end{document}
