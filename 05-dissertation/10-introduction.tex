\documentclass[00-main.tex]{subfiles}

\begin{document}

\chapter{Introduction}

\section{Background and Motivation}

Cloud computing is increasingly a major component of many computer applications.
We are experiencing a transition away from traditional native applications and towards performing the same tasks in an online environment.
However, the standard approach of building web apps with JavaScript fails to deliver the performance necessary for many intensive applications.

WebAssembly aims to solve this problem by bringing near-native performance to the browser space.
It is a virtual instruction set architecture, which executes on a stack-based virtual machine.
The WebAssembly specification has ``fast, safe, and portable semantics'' and an ``efficient and portable representation''~\ccite{wasm-spec}.

The semantics are designed to be executed efficiently across different hardware, memory safe (with respect to the surrounding execution environment), and portable across source languages, target architectures, and platforms.
The representation is compact and modular, allowing it to be efficiently transmitted over the Internet without slowing page loads.
This also includes being streamable and parallelisable, i.e.~it can be decoded while still being received.

This project creates a compiler from C to WebAssembly.
I chose C as a source language because it is reasonably low-level, allowing me to spend time working on compiler optimisations rather than just implementing language features.
In particular, C has manual memory management, therefore, I did not have to implement a garbage collector or similar.


\section{Survey of Related Work}

Many other compilers to WebAssembly exist, from a large number of source languages~\ccite{awesome-wasm-langs}.
Notably, Emscripten is a compiler to WebAssembly and JavaScript using LLVM~\ccite{emscripten-homepage,llvm}.
This primarily targets C and C++, however can also be used for any other language using LLVM\@.

Cheerp is an alternative compiler from C/C++ to WebAssembly, which can produce a combination of WebAssembly and JavaScript, and prides itself on generating high-performance code~\ccite{cheerp}.
Compilers to WebAssembly exist for other mainstream languages such as Java, C\#, Python, and even JavaScript~\ccite{cheerp-java,blazor,pyodide,javy}.
It is an area of active development; many more WebAssembly compilers are in progress for other languages~\ccite{awesome-wasm-langs}.

My project covers a subset of what the existing C to WebAssembly compilers do; I only support a subset of the language features, and focus on correctness over optimality.
I have implemented some optimisations of common cases that significantly improve the performance of many programs, but have not optimised to the extent that many industry compilers do.

\end{document}
