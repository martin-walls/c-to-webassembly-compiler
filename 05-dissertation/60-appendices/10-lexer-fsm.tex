\documentclass[../00-main.tex]{subfiles}

\begin{document}

\chapter{Lexer Finite State Machine}\label{app:lexer fsm}

In a slight abuse of regular expression notation, the dot character `\texttt{.}' and the backslash character `\texttt{\char`\\}' represent the respective characters literally.

\begin{figure}[ht]
  \centering
  \tikzfig{70-figures/10-lexer-fsm/00-numbers}
  \caption{Finite state machine for lexing number literals.}
  \label{app:fig:lexing numbers fsm} % chktex 24
\end{figure}

\begin{figure}[ht]
  \centering
  \tikzfig{70-figures/10-lexer-fsm/01-identifiers}
  \caption{Finite state machine for lexing identifiers.}
  \label{app:fig:lexing identifiers fsm} % chktex 24
\end{figure}

\begin{figure}[ht]
  \centering
  \tikzfig{70-figures/10-lexer-fsm/02-string}
  \caption{Finite state machine for lexing string literals.}
  \label{app:fig:lexing strings fsm} % chktex 24
\end{figure}

\begin{figure}[ht]
  \centering
  \tikzfig{70-figures/10-lexer-fsm/03-char}
  \caption{Finite state machine for lexing character literals.}
  \label{app:fig:lexing chars fsm} % chktex 24
\end{figure}

\begin{figure}[ht]
  \centering
  \tikzfig{70-figures/10-lexer-fsm/04-operators}
  \caption{Finite state machine for lexing operators.}
  \label{app:fig:lexing operators fsm} % chktex 24
\end{figure}

\end{document}
