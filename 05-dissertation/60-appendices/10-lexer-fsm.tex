\documentclass[../00-main.tex]{subfiles}

\begin{document}

\chapter{Lexer Finite State Machine}\label{app:lexer fsm}

\glsreset{fsm}
This chapter contains the full \gls{fsm} for the lexer (see~\ccref{sec:impl:lexer}).

Named states represent valid tokens, and blank states are invalid.
Transitions are labelled with regular expressions, matching the input character.
No input is consumed along an $\epsilon$ transition; the transition is automatically taken if the condition is true.
Only valid transitions are shown; if no transition matches the input character, either the end of the current token has been reached, or the token is invalid.
Transitions with no prior state are the initial transitions for the first input character.

If the machine stops in a named state, the corresponding token is emitted to the token stream.
However, if the machine stops in an unnamed state, the token is invalid and a lexing error is raised.

In a slight abuse of regular expression notation, the dot character~`\texttt{.}' and the backslash character~`\texttt{\char`\\}' represent the respective characters literally.

Tokens labelled as symbols represent themselves.
Named tokens correspond to the tokens described in \ccref{app:tab:token name key}.

\begin{table}[h]
  \centering
  \begin{tabular}{>{\itshape}ll} % chktex 6
    \toprule
    \normalfont\textbf{Label} & \textbf{Represented token} \\
    \midrule
    Dec & Decimal literal \\
    FP & Floating point literal \\
    Bin & Binary literal \\
    Oct & Octal literal \\
    Hex & Hexadecimal literal \\
    Iden & Identifier \\
    Keyword & C language keyword \\
    Typedef & An identifier defined as a type name \\
    String & String literal \\
    Char & Character literal \\
    \bottomrule
  \end{tabular}
  \caption{Key to token names.}%
  \label{app:tab:token name key}
\end{table}

\begin{figure}[p]
  \centering
  \tikzfig{70-figures/10-lexer-fsm/00-numbers}
  \caption{\Acrlong{fsm} for lexing number literals (decimal, floating point, binary, octal, and hexadecimal).}
  \label{app:fig:lexing numbers fsm} % chktex 24
\end{figure}

\begin{figure}[p]
  \centering
  \tikzfig{70-figures/10-lexer-fsm/01-identifiers}
  \caption{\Acrlong{fsm} for lexing identifiers, and matching them against keywords and typedef names.}
  \label{app:fig:lexing identifiers fsm} % chktex 24
\end{figure}

\begin{figure}[p]
  \centering
  \tikzfig{70-figures/10-lexer-fsm/02-string}
  \caption{\Acrlong{fsm} for lexing string literals, enclosed in double quotes.}
  \label{app:fig:lexing strings fsm} % chktex 24
\end{figure}

\begin{figure}[p]
  \centering
  \tikzfig{70-figures/10-lexer-fsm/03-char}
  \caption{\Acrlong{fsm} for lexing character literals, enclosed in single quotes. The machine handles all allowed escape sequences.}
  \label{app:fig:lexing chars fsm} % chktex 24
\end{figure}

\begin{figure}[p]
  \centering
  \tikzfig{70-figures/10-lexer-fsm/04-operators}
  \caption{\Acrlong{fsm} for lexing operators and symbols.}
  \label{app:fig:lexing operators fsm} % chktex 24
\end{figure}

\end{document}
