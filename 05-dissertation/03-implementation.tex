\documentclass[00-main.tex]{subfiles}

\begin{document}

\chapter{Implementation}

\begin{Comment}
Word budget: \textasciitilde 4500 words
\end{Comment}

\section{Repository Overview}

% - horizontal offset of vertical lines to the right
% - width of horizontal lines sticking out
% - separation between horizontal lines and start of text
% - length of horizontal lines
% - size of connecting dots
\DTsetlength{0.2em}{1em}{0.2em}{0.4pt}{2pt}

\dirtree{%
.1 /.
.2 headers\DTcomment{Stdlib definitions}.
.3 stdio.h.
.3 {...}.
.2 runtime.
.3 stdlib.
.3 init\_memory.mjs.
.3 memory\_constants.mjs.
.3 memory\_operations.mjs.
.3 number\_operations.mjs.
.3 profiler.mjs.
.3 run.mjs.
.2 src.
.3 back\_end.
.3 data\_structures.
.3 front\_end.
.3 middle\_end.
.3 program\_config.
.3 relooper.
.3 fmt\_indented.rs.
.3 id.rs.
.3 lib.rs.
.3 main.rs.
.3 preprocessor.rs.
.2 tests.
.2 tools.
}

\begin{Comment}
Finish this. Will have to see if it'll be better to have comments on the right of dirs, or to highlight the main structure below
\end{Comment}

\section{System Architecture}

\begin{Comment}
Compiler Pipeline overview -- include the diagram here
\end{Comment}

\section{Front End: Lexer and Parser}

\begin{Comment}
Describe what was actually produced.

Describe any design strategies that looked ahead to the testing phase, to demonstrate professional approach

\end{Comment}

\begin{Comment}
- wrote parser grammar

Talk about avoiding ambiguities - eg. dangling else - by using Open/Closed statement in grammar

Talk about my \texttt{interpret\_string} implementation, to handle string escaping. Implemented using an iterator.

- wrote custom lexer - cos typedefs make C context-sensitive, so handle them as we see them so they don't get mixed with identifiers

- created AST representation

Talk about structure of my AST

Talk about how I parsed type specifiers into a standard type representation. Used a bitfield to parse arithmetic types, cos they can be declared in any order.
\end{Comment}

\begin{Comment}
Describe high-level structure of codebase.

Say that I wrote it from scratch.

-> mention LALRPOP parser generator used for .lalrpop files
\end{Comment}

\section{Middle End: Intermediate Representation}

\begin{Comment}
- Defined my own three-address code representation

- for every ast node, defined transformation to 3AC instructions

- created IR data structure to hold instructions + all necessary metadata

- Talk about auto-incrementing IDs - abstraction of the Id trait and generic IdGenerator struct

- handled type information - created data structure to represent possible types

- making sure instructions are type-safe, type converting where necessary - talk about unary/binary conversions, cite the C reference book

- Compile-time evaluation of expressions, eg. for array sizes

- Talk about the Context design pattern I used throughout -- maybe research this and see if it's been done before?
\end{Comment}

\section{The Relooper Algorithm}

\begin{Comment}
cite Emscripten \cite{emscripten}
\end{Comment}

\section{Back End: Target Code Generation}


\section{Optimisations}

\subsection{Unreachable Procedure Elimination}

\subsection{Tail-Call Optimisation}

\begin{Comment}
Defn of tail-call optimisation

Why do the optimisation
\end{Comment}

\section{Summary}

\end{document}
