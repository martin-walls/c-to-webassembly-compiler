\documentclass[00-main.tex]{subfiles}

\begin{document}

\chapter{Preparation}

\begin{mrwComment}
%TC:ignore
Word budget: \textasciitilde 2500-3000 words
%TC:endignore
\end{mrwComment}

\begin{mrwComment}
%TC:ignore
Describe the work undertaken before code was written.

-> Wasm research -- include the stuff from the research doc I wrote.

-> include Relooper research here too

"Requirements Analysis" section

-> refer to appropriate software engineering techniques used in the diss

Cite new programming language learnt

Declare starting point

Explain background material required beyond IB

Researching LALRPOP - show good professional use of tools

Talk about revision control strategy, licensing of any libraries I used
%TC:endignore
\end{mrwComment}

\section{WebAssembly}

Before starting to write the compiler, I extensively researched the WebAssembly specification \ccite{wasm-spec}, to gain a deep understanding of the binary code the project aims to generate.
The following sections provide a high-level overview of the instruction set architecture.

\subsection{Primitive values}

The primitive value types supported by WebAssembly are outlined in \ccref{tab:wasm value types}, and described in more detail below.

\begin{table}[ht]
  \centering
  \begin{tabular}{lll}
    \toprule
    \textbf{Type} & \textbf{Constructor} & \textbf{Bit width} \\
    \midrule
    Integer   & \WasmType{i32}       & 32-bit \\
              & \WasmType{i64}       & 64-bit \\
    Float     & \WasmType{f32}       & 32-bit \\
              & \WasmType{f64}       & 64-bit \\
    Vector    & \WasmType{v128}      & 128-bit \\
    Reference & \WasmType{funcref}   & Opaque \\
              & \WasmType{externref} & \\
    \bottomrule
  \end{tabular}
  \caption{WebAssembly primitive types.}
  \label{tab:wasm value types}
\end{table}

Integers can be interpreted as either signed or unsigned, depending on the operations applied to them.
They will also be used to store data such as booleans and memory addresses\footnote{I.e.\ addresses within WebAssembly's sandboxed linear memory space, rather than function addresses etc..}
Integer literals are encoded in the program using the LEB128 variable-length encoding scheme \ccite{leb128-encoding}.
\Ccref{lst:leb128 pseudocode} shows how to encode an unsigned integer.
The algorithm is almost the same for signed integers (encoded in two's complement); the only difference is that we sign-extend to a multiple of 7 bits, rather than zero-extend.
When decoding, the decoder needs to know if the number is signed or unsigned, to know whether to decode it as a two's complement number or not.
This is why WebAssembly has signed and unsigned variants of some instructions.

\begin{listing}[ht]
  \begin{PseudocodeListing}
    \Function{LEB128EncodeUnsigned}{$n$}
      \State Zero-extend $n$ to a multiple of 7 bits
      \State Split $n$ into groups of 7 bits
      \State Add a 0 bit to the front of the most significant group
      \State Add a 1 bit to the front of every other group
      \State \Return bytes in little-endian order
    \EndFunction
  \end{PseudocodeListing}
  \caption{Pseudocode for the LEB128 encoding scheme (for unsigned integers).}
  \label{lst:leb128 pseudocode}
\end{listing}

Floating-point literals are encoded using the IEEE 754-2019 standard \ccite{ieee-754-2019}.
This is the same standard used by many programming languages, including Rust, so the byte representation will not need to be converted between the compiler and the WebAssembly binary.

WebAssembly also provides vector types and reference types.
Vectors can store either integers or floats: in either case, the vector is split into a number of evenly-sized numbers.
Vector instructions exist to operate on these values.
\WasmType{funcref} values are pointers to WebAssembly functions, and \WasmType{externref} values are pointers to other types of object that can be passed into WebAssembly. These would be used for indirect function calls, for example.
This project does not have a need for using either of these types; C does not have any concept of vector types, and I am not supporting function references.
I will only use the four main integer and float types.

\subsection{Instructions}

WebAssembly is a stack-based architecture; all instructions operate on the stack.
For binary instructions that take their operands from the stack, the first operand is the one that was pushed to the stack first, and the second operand is the one pushed to the stack most recently (see \ccref{lst:wasm example sub instr}).

\begin{listing}[ht]
  \begin{WasmListing}
    i32.const 10 ;; first operand
    i32.const 2  ;; second operand
    i32.sub
  \end{WasmListing}
  \caption{WebAssembly instructions to calculate \CInline{10 - 2}.}
  \label{lst:wasm example sub instr}
\end{listing}

All arithmetic instructions specify the type of value that they expect. In \ccref{lst:wasm example sub instr}, we put two \WasmType{i32} values on the stack, and use the \WasmType{i32} variant of the \WasmInstr{sub} instruction.
The module would fail to instantiate if the types did not match.
Some arithmetic instructions have signed and unsigned variants, such as the less-than instructions \WasmInstr{i32.lt_u} and \WasmInstr{i32.lt_s}.
This is the case for all instructions where the signedness of a number would make a difference to the result.

WebAssembly only supports structured control flow, in contrast to the unstructured control flow found in most instruction sets that feature arbitrary jump instructions.
There are three types of block: \WasmInline{block}, \WasmInline{loop}, and \WasmInline{if}.
The only difference between \WasmInline{block} and \WasmInline{loop} is the semantics of branch instructions.
When referring to a \WasmInline{block}, \WasmInstr{br} will jump to the end of it, and when referring to a \WasmInline{loop}, \WasmInstr{br} will jump back to the start.
This is analogous to \CInline{break} and \CInline{continue} in C, respectively.
It is worth noting that \WasmInline{loop} doesn't loop back to the start implicitly; an explicit \WasmInstr{br} instruction is required.
\WasmInline{if} blocks, which may optionally have an \WasmInline{else} block, conditionally execute depending on the value on top of the stack.
With regard to \WasmInstr{br} instructions, they behave like \WasmInline{block}.


\subsection{Modules}

A module is a unit of compilation and loading for a WebAssembly program.
There is no distinction between a `library' and a `program', such as there are in other languages; there are only modules which export functions to the instantiator.
Modules are split up into different sections.
Each section starts with a section ID and the size of the section in bytes, followed by the body of the section.
Each of the sections is described in turn below.

The \emph{type section} defines any function types used in the module, including imported functions.
Each type has a type index, allowing the same type to be used by multiple functions.

The \emph{import section} defines everything that is imported to the module from the runtime environment.
This includes imported functions as well as memories, tables, and globals.

The \emph{function section} is a map from function indexes to type indexes.
This comes before the actual body code of the functions, because this allows the module to be decoded in a single pass.
The type signature of all functions will be known before any of the code is read, so all function calls will be able to be properly typed without needing multiple passes.

A table is an array of function pointers, which can be called indirectly with \WasmInstr{call_indirect}.
It is necessary for WebAssembly to have a separate data structure for this, because functions live outside of the memory visible to the program; tables keep the function addresses opaque to the program, keeping the execution sandboxed.
The \emph{table section} stores the size limits of each table; any elements to initialise the tables with are stored in the element section.

The \emph{memory section} is similar to the table section; is specifies the size limits of each linear memory of the program\footnote{In the current WebAssembly version, only one memory is supported, and is implicitly referenced by memory instructions.}, in units of page size.

The \emph{global section} defines any global variables used in the program, including an expression to initialise them. Global variables can be either mutable or immutable.

The \emph{export section} defines everything exported from the module to the runtime environment.
Normally this is mainly function exports, however it can also include tables, memories, and global variables.

The \emph{start section} optionally specifies a function that should automatically be run when the module is initialised.
This could be used to initialise a global variable to a non-constant expression.

The \emph{element section} is used to statically initialise the contents of tables with function addresses.
Each element segment within this section starts with a flag that specifies the mode of the segment.
For example, one mode is to automatically initialise the elements to table 0 upon module initialisation.

The \emph{data count section} is an optional additional section used to allow validators to use only a single pass.
It specifies how many data segments are in the data section.
If the data count section were not present, a validator would not be able to check the validity of instructions that reference data indexes until after reading the data section; but because this comes after the code section, it would require multiple passes over the module.
This section has no effect on the actual execution.

The \emph{code section} contains the instructions for each function body.
All code in a WebAssembly program is contained in a function.
Each function body begins by declaring any local variables, followed by the code for that function.

The \emph{data section} is similar to the element section; it is used to statically initialise the contents of memory.
This can be used, for example, to load string literals into memory.


\section{The Relooper Algorithm}

C allows arbitrary control flow, because it has \CInline{goto} statements.
However, WebAssembly only has structured control flow, using \WasmInline{block}, \WasmInline{loop}, and \WasmInline{if} constructs as described above.
Therefore, once the intermediate code has been generated, it needs to be transformed to only have structured control flow.

There are several algorithms that achieve this.
The most naive solution would be to use a label variable and one big \CInline{switch} statement containing all the basic blocks of the program; the label variable is set at the end of each block, and determines which block to switch to next. However, this is very inefficient.

The first algorithm that solved this problem in the context of compiling to WebAssembly (and JavaScript, before that) was the Relooper algorithm, introduced by Emscripten in their paper on compiling LLVM to JavaScript \ccite{emscripten}.
In today's WebAssembly compilers, there are three general methods used to convert to structured control flow \ccite{wasm-control-flow-restructuring-thread, solving-structured-control-flow-problem}: Emscripten/Binaryen's Relooper algorithm, LLVM's CFGStackify, and Cheerp's Stackifier.
All of these implementations, including the modern implementation of Relooper, are more optimised than the original Relooper algorithm, however therefore also more complex.
The original Relooper algorithm is a greedy algorithm, and is well described in the paper, and is the one I decided to implement.

\subsection{Input and Output}

The Relooper algorithm takes a so-called `soup of blocks' as input.
Each block is a basic block of the flowgraph, that begins at an instruction label and ends with a branch instruction (\ccref{fig:relooper input label structure}).
There can be no labels or branch instructions other than at the start or end of the block.
The branch instruction may either be a conditional branch (i.e.\ \CInline{if (...) goto x else goto y}), or an unconditional branch.

To avoid overloading the term \emph{block}, these input blocks are referred to as \emph{labels}.


\begin{figure}[ht]
  \centering
  \scalebox{0.9}{\tikzfig{70-figures/03-relooper-input-block}}
  \caption{Structure of labels (Relooper algorithm input blocks).}
  \label{fig:relooper input label structure}
\end{figure}

The algorithm generates a set of structured blocks, recursively nested to represent the control flow (\ccref{fig:relooper output blocks structure}).
\emph{Simple} blocks represent linear control flow; they contain an \emph{internal} label, which contains the actual program instructions.
\emph{Loop} blocks contain an \emph{inner} block, which contains all the labels that can possibly loop back to the start of the loop, along some execution path.
\emph{Multiple} blocks represent conditional execution, where execution can flow along one of several paths.
They contain \emph{handled} blocks to which execution can pass when we enter the block.
All three blocks have a \emph{next} block, where execution will continue.

\begin{figure}[ht]
  \centering
  \scalebox{0.9}{\tikzfig{70-figures/04-relooper-output-blocks}}
  \caption{Structured blocks, generated by the Relooper algorithm.}
  \label{fig:relooper output blocks structure}
\end{figure}

\subsection{Algorithm}

Before describing the algorithm, we define the terms we used.
\emph{Entries} are the subset of labels that can be immediately reached when a block is entered.
Each label has a set of \emph{possible branch targets}, which are the labels it directly branches to.
It also has a set of labels it can \emph{reach}, known as the \emph{reachability} of the label.
This is the transitive closure of the possible branch targets; i.e.\ all labels that can be reached along some execution path.
Labels can always reach themselves.

Given a set of labels, and set of entries, the algorithm to create a block is as follows:

\begin{enumerate}[label=\boxed{\arabic*}]
\item % (1)
Calculate the reachability of each label.
\item % (2)
If there is only one entry, and execution cannot return to it, create a simple block with the entry as the internal label.
\begin{itemize}
\item
Construct the next block from the remaining labels; the entries for the next block are the possible branch targets of the internal label.
\end{itemize}
\item % (3)
\label[algstep]{alg:relooper:create a loop block}
If execution can return to every entry along some path, create a loop block.
\begin{itemize}
\item
Construct the inner block from all labels that can reach at least one of the entries.
\item
Construct the next block from the remaining labels; the entries for the next block are all the labels in the next block that are possible branch targets of labels in the inner block.
\end{itemize}

\item % (4)
\label[algstep]{alg:relooper:attempt to create multiple block}
If there is more than one entry, attempt to create a multiple block. (This may not be possible.)
\begin{itemize}
\item
For each entry, find any labels that it reaches that no other entry reaches.
If any entry has such labels, it is possible to create a multiple block.
If not, go to \ccref{alg:relooper:create loop block after multiple block failed}.
\item
Create a handled block for each entry that uniquely reaches labels, containing all those labels.
\item
Construct the next block from the remaining labels. The entries are the remaining entries we didn't create handled blocks from, plus any other possible branch targets out of the handled blocks.
\end{itemize}

\item % (5)
\label[algstep]{alg:relooper:create loop block after multiple block failed}
If \ccref{alg:relooper:attempt to create multiple block} fails, create a loop block in the same way as \ccref{alg:relooper:create a loop block}.
\end{enumerate}

The Emscripten paper outlines a proof that this greedy approach will always succeed \ccite[p. 10]{emscripten}.
The core idea is that we can show that (1) whenever the algorithm terminates, the block it outputs is correct with respect to the original program semantics; (2) the problem is strictly simplified every time a block is created, therefore the algorithm must terminate; and (3) the algorithm is always able to create a block from the input labels.
Point (3) lies in the observation that if we reach \ccref{alg:relooper:create loop block after multiple block failed}, we must be able to create a loop block.
This holds because if we could not create a loop block, we would not be able to return to any of the entries; however, in that case it would be possible to create a multiple block in \ccref{alg:relooper:attempt to create multiple block}, because the entries would uniquely reach themselves.

The algorithm replaces the branch instructions in the labels as it processes them.
When creating a loop block, branch instructions to the start of the loop are replaced with a \CInline{continue}, and any branches to outside the loop are replaced with a \CInline{break} (all the branch targets outside the loop will be in the next block).
When creating a handled block, branch instructions into the next block are replaced with an \CKeyword{endHandled} instruction.
Each of these instructions is annotated with the ID of the block they act on, because block may be nested.
Note that functionally, an \CKeyword{endHandled} instruction is equivalent to a \CInline{break}, but I chose to keep the two distinct because they store IDs of different block types.

To direct control flow when execution enters a multiple block, Relooper makes use of a label variable.
Whenever a branch instruction is replaced, an instruction is inserted to set the label variable to the label ID of the branch target.
Handled blocks are executed conditional on the value of the label variable.
This will generate some overhead of unnecessary instructions, since most of the time the label variable is set, it will not be checked.
However, later stages of the compiler pipeline are able to optimise this away.


\section{Rust}


\section{Project Strategy}

\subsection{Requirements Analysis}

\subsection{Software Engineering Methodology}

\subsection{Testing}

\section{Starting Point}

\subsection{Knowledge and experience}

\begin{mrwComment}
%TC:ignore
- IB Compilers Course

- Experience with JavaScript + Python (cos I used those for runtime/testing)

- Experience writing C, my source language
%TC:endignore
\end{mrwComment}

\subsection{Tools Used}

\begin{mrwComment}
%TC:ignore
- Say here that I learned Rust for this project -- talk about the borrow checker and memory safety

- Also used JavaScript for runtime + Python for testing
%TC:endignore
\end{mrwComment}


\end{document}
