\documentclass[00-main.tex]{subfiles}

\begin{document}

\chapter{Evaluation}

\newcommand{\IncludeStackPlot}[1]{\resizebox{0.98\textwidth}{!}{\input{71-plots/#1}}}

\begin{Comment}
%TC:ignore
Word budget: \textasciitilde 2000--2400 words
%TC:endignore
\end{Comment}

\begin{Comment}
%TC:ignore
``Signs of success, evidence of thorough and systematic evaluation''

- How many of the original goals were achieved?

- Were they proved to have been achieved?

- Did the program really work?

- Answer questions posed in the introduction

- use appropriate techniques for evaluation, eg. confidence intervals
%TC:endignore
\end{Comment}


In this chapter, I evaluate my project against my success criteria, showing that all the success criteria were achieved.
In \ccref{sec:eval:correctness testing} I demonstrate that the compiler is correct using a variety of test programs.
In \ccref{sec:eval:optimisations} I evaluate the impact of the optimisations I implemented, particularly showing significant improvements in memory usage.


\section{Success Criteria}

The success criteria for my project, as defined in my project proposal, are:
\begin{itemize}
\item The program generates an abstract syntax tree from C source code.
\item The program transforms the abstract syntax tree into an intermediate representation.
\item The program uses the Relooper algorithm to transform unstructured to structured control
flow.
\item The program generates WebAssembly binary code from the intermediate representation.
\item The compiler generates binary code that produces the same output as the source program.
\end{itemize}

All of my success criteria have been met.
The first four criteria correspond to the main stages of the compiler pipeline, respectively.
The correctness of this pipeline is verified by the correctness of the generated binary code.
The generated binary would not be correct if any of the stages had not been successful.

I used a variety of test programs to verify the success of the fifth criteria. This is described in \ccref{sec:eval:correctness testing}.


\section{Correctness Testing} \label{sec:eval:correctness testing}

I wrote a suite of test programs in C to evaluate the correctness and performance of my compiler.
There were two types of program:
\begin{itemize}
\item 18 `unit test' programs, which test a specific construct in the C language; and
\item 11 `full programs', which represent real workloads that the compiler would be used for.
\end{itemize}

Programs of the first type are not strictly unit tests by the standard definition.
Unit tests verify the functionality of individual units of source code, in isolation from the rest of the application.
Each test should test one particular behaviour of that unit, and should be independent from the rest of the program's functionality \ccite{unit-testing}.
My test programs don't test an isolated behaviour of the compiler's source code.
Instead, they test a single behaviour of the C source code that is being compiled (for example, dereferencing a pointer), verifying that the compiler pipeline maintains the correct behaviour.
This allowed me to trace bugs to the units of code that transformed that particular construct.

Examples of `full programs' include Conway's Game of Life \ccite{conways-game-of-life}, calculating the Fibonacci numbers (recursively), and finding occurrences of a substring in a string.
Six of the test programs I used were sourced from Clib \ccite{clib}, which contains many small utility packages for C. Those that I used had no external dependencies, and were useful in verifying that my compiler worked for other people's code as well as my own. Clib is licensed under the MIT license, which permits use of the software ``without restriction''.
The remaining five full programs and all 18 unit test programs were written by myself.

For all my tests, I used GCC\footnote{GCC version 11.3.1.} as the reference for correctness.
I deemed a program to be correct if it produced the same output as when compiled with GCC.
To facilitate this, I made liberal use of \texttt{printf}, to output the results of computations.

I wrote a test runner script to ensure that I maintained correctness as I continued to develop the compiler, fix bugs, and implement optimisations.
This script read a directory of \Filename{.yaml} files, which described the path to each test program, and the arguments to run it with.
It then compiled the program with both my compiler and with GCC, and compared the outputs.
A test passed if the outputs were identical, and failed otherwise.
The script reported which tests, if any, failed.

I also implemented some convenience features into the test script.
The command-line interface takes an optional `filter' argument, which can be used to run a subset of the tests whose name matches the filter.
The script can also be used to run one of the test programs without comparing to GCC, printing to the standard output. This allows easier manual testing.

To prevent bugs from accidentally being introduced into my compiler, I set up the test suite to run automatically as a commit hook whenever I committed changes to the Git repository.
This would prevent a commit from succeeding if any of the tests failed, allowing me to make corrections first.
This ensured that the version of my project in source control was always correct and functioning.




\section{Impacts of Optimisations} \label{sec:eval:optimisations}

In the following sections, I will evaluate the impacts that the optimisations I implemented had.
For unreachable procedure elimination, I will evaluate the reduction in code size.
For tail-call optimisation and stack allocation optimisation, I will evaluate the effectiveness at reducing memory usage.

To evaluate the memory usage optimisations, I inserted profiling code when compiling the programs, to measure the size of the stack throughout program execution.
When generating instructions that move the stack pointer, the compiler additionally inserts a call to \JSInlineCode{log_stack_ptr}, a function imported from the JavaScript runtime.
\JSInlineCode{log_stack_ptr} reads the current stack pointer value from memory and appends it to a log file.
I wrote a Python script to visualise the resulting data.
The plots show how the size of the stack grows and shrinks throughout the execution of the program.
\Ccref{fig:tail-call optimisation stack use,fig:comparing stack usage optimisation across programs,fig:comparing stack usage for case.c,fig:comparing stack usage for fibonacci.c} are generated using this method.

\subsection{Unreachable Procedure Elimination}

In the context of this project, unreachable procedure elimination mainly has benefits in removing unused standard library functions from the compiled binary.
When a standard library header is included in the program, the preprocessing stage inserts the entire code of that library\footnote{That is, the entirety of my skeleton implementation of that library, rather than the actual standard library code.}.
If the program only uses one or two of the functions, most of them will be redundant.
Unreachable procedure elimination is able to safely remove these, resulting in a smaller binary.

I only implemented enough of the standard library to allow my test programs to run, so the impact of this optimisation is limited by the number of functions imported.
If I were to implement more of the standard library, this optimisation would become more important.

The standard library header with the most functions that I implemented was \Filename{ctype.h}.
This header contains 13 functions, of which a program might normally use two or three.
This is where I saw the biggest improvement from the optimisation.
Programs that used the \CInlineCode{ctype} library saw an average file-size reduction of \SI{4.7}{\kilo\byte}.

The other standard library headers I implemented only contained a few functions, so the impact of this optimisation was much more limited. However, as mentioned above, if I implemented more of the standard library, I would see much more of an improvement.

Due to this difference in the standard library header files, and also to the fact that source programs can arbitrarily contain functions that are never used, it is not meaningful to calculate aggregate metrics across all my test programs.
However, testing each program individually does verify that any functions that are unused are removed from the compiled binary.

\begin{Comment}
%TC:ignore
Can you graph the results from this section?  Benchmarks along the x
axis and code size before / after optimisation on the y axis.  Or
reduction in code size on y?
%TC:endignore
\end{Comment}


\subsection{Tail-Call Optimisation}

My implementation of tail-call optimisation\index{Tail-call optimisation} was successful in reusing the existing function stack frame for tail-recursive calls. The recursion is converted into iteration within the function, eliminating the need for new stack frame allocations.
Therefore, the stack memory usage remains constant rather than growing linearly with the number of recursive calls.

One of the functions I used to evaluate this optimisation was the function in \ccref{lst:tail-recursive sum} below, that uses tail-recursion to compute the sum of the first $n$ integers.

\begin{listing}[ht]
  \CInputListing[firstline=4, lastline=9]{../02-test-programs/15-tail-call-optimisation/02-tail-call-sum.c}
  \caption{Tail-recursive function to sum the integers 1 to $n$}
  \label{lst:tail-recursive sum}
\end{listing}


\Ccref{fig:tail-call optimisation stack use} compares the stack memory usage with tail-call optimisation disabled and enabled.
Without the optimisation, the stack size clearly grows linearly with $n$.
When running the program with $n=500$, a stack size of \SI{46.3}{\kilo\byte} is reached.
When the same program is compiled with tail-call optimisation enabled, only 298 bytes of stack space are used; a \SI{99.36}{\percent} reduction in memory usage.
Of course, the reduction depends on how many iterations of the function are run.
In the non-optimised case, the stack usage is \bigo{n} in the number of iterations, whereas in the optimised case it is \bigo{1}.

\begin{figure}[ht]
  \centering
  \IncludeStackPlot{22-tailcall-sum-compare-tailcallopt-without-stackopt.pgf}
  \caption{Stack usage for calling \texttt{sum(500, 0)} (see \ccref{lst:tail-recursive sum})}
  \label{fig:tail-call optimisation stack use}
\end{figure}

When testing with large $n$, the non-optimised version quickly runs out of memory space, and throws an exception.
In contrast, the optimised version has no memory constraint on how many iterations can be run.
It successfully runs \num{1000000} levels of recursion without using any more memory than for small $n$; even GCC fails to run that many.


\subsection{Optimised Stack Allocation Policy}

The stack allocation policy that I implemented was successful in reducing the amount of stack memory used.

From my test programs, the largest reduction in memory use was \SI{69.73}{\percent} compared to the unoptimised program.
The average improvement was \SI{50.28}{\percent}.
% with a \SI{95}{\percent}~confidence interval of $[41.37\%, 59.19\%]$.

\Ccref{fig:comparing stack usage optimisation across programs} shows the impact that this optimisation had on the different test programs.
The full-height bars represent the stack usage of the unoptimised program, which we measure the optimised program against.
The darker bars show the stack usage of the optimised program as a percentage of the original stack usage.
Shorter bars represent a greater improvement (less memory is being used).

For programs that benefit from tail-call optimisation, I measured the effect of this optimisation on both the optimised and unoptimised versions.
I did this because tail-call optimisation also affects how much memory is used, so it may have an impact on the effectiveness of this optimisation.

\begin{figure}[ht]
  \centering
  \adjustbox{clip, trim=0cm 1.2cm 0cm 0cm}{\IncludeStackPlot{40-stack-use-across-programs.pgf}}
  \caption{Comparing optimised stack usage to unoptimised stack usage. Shorter bars represent greater improvement.}
  \label{fig:comparing stack usage optimisation across programs}
\end{figure}

One of the main factors influencing the amount of improvement is the number of temporary variables generated.
The more temporary variables generated, the larger each stack frame will be in the unoptimised version, and the more scope there is for the compiler to find non-clashing variables to overlap.
Because temporary variables are generated locally for each instruction, the majority of them only have short-range dependencies.
Only the variables that correspond to user variables have longer-range dependencies.
Therefore the temporary variables offer the compiler more options of independent variables.

The result of this is that as a function increases in its number of operations, the number of temporary variables increases, and so does the scope for optimisation that the compiler is able to exploit.

We can see the effect of this directly when we compare the stack usage of the unoptimised and optimised versions of the same program.
\Ccref{fig:comparing stack usage for case.c} shows the size of the stack over the execution of a test program that converts strings to upper, lower, or camel case.
Since the main stack allocations and deallocations occur on function calls and returns respectively, each spike on the plot corresponds to a function call. We can use this to figure out which parts of the plot correspond to which part of the source program.

\begin{figure}[ht]
  \centering
  \IncludeStackPlot{01-case-compare.pgf}
  \caption{Comparing stack usage for \Filename{case.c}.}
  \label{fig:comparing stack usage for case.c}
\end{figure}

The program in turn calls \CInlineCode{case_upper()}, \CInlineCode{case_lower()}, and \CInlineCode{case_camel()}, which corresponds to the three distinct sections of the plot.

\CInlineCode{case_upper()} and \CInlineCode{case_lower()} each make repeated calls to \CInlineCode{toupper()} and \CInlineCode{tolower()}, which corresponds to the many short spikes on the plot (one for each character in the string).
Other than a \CInlineCode{for} loop, they do not contain many operations, and therefore not many temporary variables are generated.


In contrast, \CInlineCode{case_camel()} performs many more operations iteratively in the body of the function.
\Ccref{lst:section of casecamel body code} shows an extract of its body code.
Even in this short section, more temporary variables are created than in the entire body of \CInlineCode{case_upper()}.
This results in the large spike at the end of \ccref{fig:comparing stack usage for case.c}.

\begin{listing}[!ht]
  \CInputListing[firstline=18, lastline=21]{../02-test-programs/12-single-source-programs/case.c}
  \caption{The entire body of \CInlineCode{case_upper()}.}
  \label{lst:caseupper body code}
\end{listing}

\begin{listing}[!ht]
  \CInputListing[firstline=39, lastline=43]{../02-test-programs/12-single-source-programs/case.c}
  \caption{A short section of the body of \CInlineCode{case_camel()}.}
  \label{lst:section of casecamel body code}
\end{listing}

Due to the differences in temporary variables described above, the compiler is able to optimise \CInlineCode{case_camel()} much more than the other functions.
This parallels the fact that \CInlineCode{case_camel()} had the largest stack frame initially.


Another area that this optimisation has a large impact is for recursive functions.
Since this optimisation reduces the size of each stack frame, we will see a large improvement when we have lots of recursive stack frames.
\Ccref{fig:comparing stack usage for fibonacci.c} shows the size of the stack over the execution of calculating the Fibonacci numbers recursively.
In this instance, the optimised stack allocation policy reduced the stack size of the program by \SI{67.20}{\percent}.


\begin{figure}[ht]
  \centering
  \IncludeStackPlot{02-fibonacci-compare.pgf}
  \caption{Comparing stack usage for \Filename{fibonacci.c}.}
  \label{fig:comparing stack usage for fibonacci.c}
\end{figure}



\section{Summary}

The main objective of this project was to produce a compiler that generated correct WebAssembly binary code.
Through the range of testing described above, I have shown that this objective was achieved, with the definition of correctness being that the generated program behaves in the same was as when compiled with GCC.

The objective of adding optimisations is to improve the performance of the compiled programs, while maintaining the semantic meaning of the program.
The correctness of my optimisations was verified with the same test suite as was used to test the unoptimised compiler's correctness.
In the previous sections, we have seen measurable evidence that the optimisations did improve performance.
Therefore the optimisations were a success.


\end{document}
